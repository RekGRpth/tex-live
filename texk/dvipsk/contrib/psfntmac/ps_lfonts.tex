% File LFONTS - Version of 21 November 1989,
% plus modifications for PostScript Times family fonts (added by Larry
% Denenberg, larry@bbn.com, originally 5 Apr 88 and later 13 Feb 91).
% PS font names changed to use Karl Berry's names like dvips does,
%  Stephen Gildea, Sep 92.

% This version of LFONTS.TEX is for the CMR fonts.  It was converted
% from the AMR version by David Fuchs on 18 December 1985.
% And corrected on 11 Nov 1986 by Leslie Lamport.
% Last vestige of AMR fonts removed 3 Mar 89 by Leslie Lamport.

% This file needs to be customized for the fonts available at a particular
% site.  There are three places where changes need to be made.  They
% can be found by searching this file for the string  FONT-CUSTOMIZING.
%
% FONT CONVENTIONS
%
% A TYPESTYLE COMMAND is something like \it that defines a type style.
% Each style command \xx is defined to be \protect\pxx, where
% \pxx is defined to choose the correct font for the current size.
% This allows style commands to appear in 'unsafe' arguments where
% protection is required.
%
% A SIZE COMMAND is something like \normalsize that defines a type size.
% It is defined by the document style.  However, \normalsize is handled
% somewhat differently because it is called so often--e.g., on every
% page by the output routine.  The document style defines \@normalsize
% instead of \normalsize.
%
% A ONE-SIZE typestyle is one that exists only in the \normalsize size.
%
% A FONT-SIZE COMMAND is one that defines \textfont, \scriptfont and
% \scriptscriptfont for the font families corresponding to preloaded fonts,
% as well as the typestyle commands for the preloaded fonts.  Each
% font-size command has an associated @fontsize command, having the same
% name except for an '@' at the front.   All font-size commands are defined
% in LFONTS.  The naming convention is that a fifteenpt font has a font-size
% name \xvpt, and so on.
%
% Each size command \SIZE executes the command
%             \@setsize\SIZE{BASELINESKIP}\FONTSIZE\@FONTSIZE
% which does the following.
%   0. Executes \@nomath\SIZE to issue warning if in math mode.
%   1. \let \@currsize = \SIZE
%   2. Sets \strutbox to a strut of height .7 * BASELINESKIP and
%      depth .3 * BASELINESKIP
%
%       Note: Charles Karney observed that step 2 is useless, since the
%       \FONTSIZE command executed in step 4 resets \strutbox using
%       the actual baselineskip, which is \baselinestretch * BASELINESKIP.
%       Some day, this step may get removed.  (Note made 28 Feb 89)
%
%   3. Sets \baselineskip to \baselinestretch * BASELINESKIP
%      and
%   4. Calls \FONTSIZE
%   5. Executes the \@FONTSIZE command.
% It should then define all the typestyle commands not defined by the font-size
% command, except for the one-size type styles. A typestyle command for which
% the corresponding font exists but is not preloaded is defined to expand to a
% \@getfont command.  A typestyle whose font does not exist is defined to
% expand to a \@subfont command.
%
% A one-size typestyle whose font is not preloaded is defined to expand to
% a \@onesizefont command.
%
% \em is defined to be \it inside an unslanted style and \rm inside a
% slanted style.  An \em command in a section title will produce a \pem
% command in the table of contents.
%

\def\em{\protect\pem{}}
\def\pem{\ifdim \fontdimen\@ne\font >\z@ \rm \else \it \fi}

\def\normalsize{\ifx\@currsize\normalsize \rm \else \@normalsize\fi}

% \load{SIZE}{STYLE} : Solves anomaly of loaded-on-demand font
%    used for first time in math mode.  Give this command outside math
%    mode, before formula using it for first time.
\def\load#1#2{\let\@tempa\@currsize \let\@currsize\relax #1#2\@tempa}

% \newfont{\CMD}{FONT} defines \CMD to be the font FONT.
%    It is equivalent to \font \CMD = FONT
% \symbol{NUM} == \char NUM

\def\newfont#1#2{\@ifdefinable #1{\font #1=#2\relax}}
\def\symbol#1{\char #1\relax}



% \@getfont \STYLE \FAM \@FONTSIZE{LOADING.INFO}
%   \STYLE       = style command
%   \FAM         = a control sequence defined by \newfam\FAM
%   \@FONTSIZE   = the @fontsize command for the current size.
%   LOADING.INFO = information needed to load the font--e.g.,
%                  cmtti10 \magstep 2
%   Does the following, where \FONTNAME denotes a new unique, untypeable
%   font name:
%    1. Executes  \font \FONTNAME = LOADING.INFO
%    2. Appends '\textfont FAM \FONTNAME \def\STYLE{\fam \FAM \FONTNAME}'
%       to the definition of \@FONTSIZE.
%    3. Executes \@FONTSIZE \STYLE.
%
% \@nohyphens\STYLE\@FONTSIZE
%   Used right after \@getfont to set \hyphenchar of the new font to -1,
%   thereby prohibiting hyphenation.  It is used with \tt fonts.
%   (\@nohyphens was added on 12/18/85)
%
% \@subfont \STYLE \REPSTYLE
%   \STYLE, \REPSTYLE = type style commands.
%   Types warning message and defines uses \REPSTYLE.
%
% \@onesizefont \STYLE {LOADING.INFO}
%   Defines \STYLE to be a typestyle that exists only for the \normalsize
%   size.  It produces the font specified by LOADING.INFO
%
% \@addfontinfo\@FONTSIZE{DEFS}
%    Expands DEFS and adds to the definition of \@FONTSIZE. Items that should
%    not be expanded should be protected with \@prtct---except no protection
%    is needed for '\def\foo', only for the contents of the definition.
%
% \@nomath\CS : Types a warning '\CS used in math mode' if encountered
%    in math mode.

% Remove \outer from definition of \newfam
\def\newfam{\alloc@8\fam\chardef\sixt@@n}

\def\@setsize#1#2#3#4{\@nomath#1\let\@currsize#1\baselineskip
   #2\setbox\strutbox\hbox{\vrule height.7\baselineskip
      depth.3\baselineskip width\z@}\baselineskip\baselinestretch\baselineskip
   \normalbaselineskip\baselineskip#3#4}

\newif\if@bold

\let\@prtct=\relax

\def\@addfontinfo#1#2{{\def\@prtct{\noexpand\@prtct\noexpand}\def\def{\noexpand
    \def\noexpand}\xdef#1{#1#2}}}

\def\@getfont#1#2#3#4{\@ifundefined{\string #1\string #3}{\global\expandafter
    \font \csname \string #1\string #3\endcsname #4\relax
     \@addfontinfo#3{\textfont #2\csname \string #1\string #3\endcsname
     \scriptfont #2\csname \string #1\string #3\endcsname
     \scriptscriptfont #2\csname \string #1\string #3\endcsname
     \def#1{\fam #2\csname\string #1\string #3\endcsname}}}{}#3#1}

\def\@nohyphens#1#2{\global\expandafter \hyphenchar\csname
   \string #1\string #2\endcsname \m@ne}

\def\@subfont#1#2{\@warning{No \string#1\space typeface in
        this size, using \string#2}#2}

\def\@onesizefont#1#2{\expandafter\newfam\csname fm\string#1\endcsname
  \global\expandafter\font\csname ft\string#1\endcsname #2\relax
  \gdef#1{\ifx \@currsize\normalsize \@ftfam#1\else
  \@warning{Typeface \string#1\space available only in
  \string\normalsize, using \string\rm}\gdef #1{\ifx \@currsize\normalsize
  \textfont\@fontfam#1 \scriptfont\@fontfam#1 \scriptscriptfont
  \@fontfam#1\@ftfam#1\else \rm\fi}#1\fi}#1}

\def\@ftfam#1{\fam\csname fm\string#1\endcsname\csname ft\string#1\endcsname}

\def\@nomath#1{\ifmmode \@warning{\string#1\space in math mode.}\fi}
\def\@nomathbold{\ifmmode \@warning{\string\mathbold\space in math mode.}\fi}

% The following definitions save token space.  E.g., using \@height
% instead of height saves 5 tokens at the cost in time of one macro
% expansion.

\def\@height{height}
\def\@depth{depth}
\def\@width{width}

\def\@magscale#1{ scaled \magstep #1}
\def\@halfmag{ scaled \magstephalf}
\def\@ptscale#1{ scaled #100}


%% FONT-CUSTOMIZING:  The following \font commands define the
%% preloaded LaTeX fonts.  Font names should be changed to cause
%% different fonts to be loaded in place of these particular AMR fonts.
%% \font commands should be added or deleted to change which fonts
%% are preloaded.

% five point
 \font\fivrm  = ptmr at 5pt               % roman
 \font\fivmi  = cmmi5              % math italic
    \skewchar\fivmi ='177          %  for placement of accents
%\font\fivmib = cmmib10  \@ptscale5 % bold math italic
 \font\fivsy  = cmsy5              % math symbols
    \skewchar\fivsy ='60           %   for placement of math accents
%\font\fivsyb = cmbsy10 \@ptscale5 % bold math symbols
%\font\fivit  = ptmri at 5pt       % text italic
%\font\fivsl  = ptmro at 5pt       % slanted
%\font\fivbf  = ptmb at 5pt       % extended bold
%\font\fivbfs = ptmbo at 5pt    % extended bold slanted
%\font\fivtt  = pcrb at 5pt       % typewriter
%\font\fivtti = cmitt10 \@ptscale5 % italic typewriter
%\font\fivtts = pcrbo at 5pt       % slanted typewriter
%\font\fivsf  = phvr at 5pt       % sans serif
%\font\fivsfi = phvro at 5pt       % italic sans serif
%\font\fivsfb = phvb at 5pt       % bold sans serif
%\font\fivsc  = cmcsc10 \@ptscale5 % small caps
 \font\fivly  = lasy5             % LaTeX symbols
%\font\fivlyb = lasyb10 \@ptscale5 % LaTeX symbols
%\font\fivuit = ptmru at 5pt     % unslanted italic

% six point
 \font\sixrm  = ptmr at 6pt       % roman
 \font\sixmi  = cmmi6              % math italic
    \skewchar\sixmi ='177          %  for placement of accents
%\font\sixmib = cmmib10  \@ptscale6 % bold math italic
 \font\sixsy  = cmsy6              % math symbols
    \skewchar\sixsy ='60           %   for placement of math accents
%\font\sixsyb = cmbsy10 \@ptscale6 % bold math symbols
%\font\sixit  = ptmri at 6pt       % text italic
%\font\sixsl  = ptmro at 6pt       % slanted
%\font\sixbf  = ptmb at 6pt       % extended bold
%\font\sixbfs = ptmbo at 6pt    % extended bold slanted
%\font\sixtt  = pcrb at 6pt       % typewriter
%\font\sixtti = cmitt10 \@ptscale6 % italic typewriter
%\font\sixtts = pcrbo at 6pt       % slanted typewriter
%\font\sixsf  = phvr at 6pt       % sans serif
%\font\sixsfi = cmssi10 \@ptscale6 % italic sans serif
%\font\sixsfb = phvb at 6pt       % bold sans serif
%\font\sixsc  = ptmrc at 6pt     % small caps
 \font\sixly  = lasy6             % LaTeX symbols
%\font\sixlyb = lasyb10 \@ptscale6 % LaTeX symbols
%\font\sixuit = ptmru at 6pt     % unslanted italic

% seven point
 \font\sevrm  = ptmr at 7pt       % roman
 \font\sevmi  = cmmi7              % math italic
    \skewchar\sevmi ='177          %  for placement of accents
%\font\sevmib = cmmib10  \@ptscale7 % bold math italic
 \font\sevsy  = cmsy7              % math symbols
    \skewchar\sevsy ='60           %   for placement of math accents
%\font\sevsyb = cmbsy10 \@ptscale7 % bold math symbols
 \font\sevit  = ptmri at 7pt       % text italic
%\font\sevsl  = ptmro at 7pt       % slanted
%\font\sevbf  = ptmb at 7pt       % extended bold
%\font\sevbfs = ptmbo at 7pt    % extended bold slanted
%\font\sevtt  = pcrb at 7pt       % typewriter
%\font\sevtti = cmitt10 \@ptscale7 % italic typewriter
%\font\sevtts = pcrbo at 7pt       % slanted typewriter
%\font\sevsf  = phvr at 7pt       % sans serif
%\font\sevsfi = cmssi10 \@ptscale7 % italic sans serif
%\font\sevsfb = phvb at 7pt       % bold sans serif
%\font\sevsc  = ptmrc at 7pt     % small caps
 \font\sevly  = lasy7             % LaTeX symbols
%\font\sevlyb = lasyb10 \@ptscale7 % LaTeX symbols
%\font\sevuit = ptmru at 7pt     % unslanted italic

% eight point
 \font\egtrm  = ptmr at 8pt       % roman
 \font\egtmi  = cmmi8              % math italic
    \skewchar\egtmi ='177          %  for placement of accents
%\font\egtmib = cmmib10  \@ptscale8 % bold math italic
 \font\egtsy  = cmsy8              % math symbols
    \skewchar\egtsy ='60           %   for placement of math accents
%\font\egtsyb = cmbsy10 \@ptscale8 % bold math symbols
 \font\egtit  = ptmri at 8pt       % text italic
%\font\egtsl  = ptmro at 8pt       % slanted
%\font\egtbf  = ptmb at 8pt       % extended bold
%\font\egtbfs = ptmbo at 8pt    % extended bold slanted
%\font\egttt  = pcrb at 8pt       % typewriter
%\font\egttti = cmitt10 \@ptscale8 % italic typewriter
%\font\egttts = pcrbo at 8pt       % slanted typewriter
%\font\egtsf  = phvr at 8pt       % sans serif
%\font\egtsfi = cmssi10 \@ptscale8 % italic sans serif
%\font\egtsfb = phvb at 8pt       % bold sans serif
%\font\egtsc  = ptmrc at 8pt     % small caps
 \font\egtly  = lasy8             % LaTeX symbols
%\font\egtlyb = lasyb10 \@ptscale8 % LaTeX symbols
%\font\egtuit = ptmru at 8pt     % unslanted italic


% nine point
 \font\ninrm  = ptmr at 9pt       % roman
 \font\ninmi  = cmmi9              % math italic
    \skewchar\ninmi ='177          %  for placement of accents
%\font\ninmib = cmmib10  \@ptscale9 % bold math italic
 \font\ninsy  = cmsy9              % math symbols
    \skewchar\ninsy ='60           %   for placement of math accents
%\font\ninsyb = cmbsy10 \@ptscale9 % bold math symbols
 \font\ninit  = ptmri at 9pt       % text italic
%\font\ninsl  = ptmro at 9pt       % slanted
 \font\ninbf  = ptmb at 9pt       % extended bold
%\font\ninbfs = ptmbo at 9pt    % extended bold slanted
 \font\nintt  = pcrb at 9pt       % typewriter
    \hyphenchar\nintt = -1         %  suppress hyphenation in \tt font
%\font\nintti = cmitt10 \@ptscale9 % italic typewriter
%\font\nintts = pcrbo at 9pt       % slanted typewriter
%\font\ninsf  = phvr at 9pt       % sans serif
%\font\ninsfi = cmssi10 \@ptscale9 % italic sans serif
%\font\ninsfb = phvb at 9pt       % bold sans serif
%\font\ninsc  = ptmrc at 9pt     % small caps
 \font\ninly  = lasy9             % LaTeX symbols
%\font\ninlyb = lasyb10 \@ptscale9 % LaTeX symbols
%\font\ninuit = ptmru at 9pt     % unslanted italic

% ten point
 \font\tenrm  = ptmr at 10pt    % roman
 \font\tenmi  = cmmi10   % math italic
    \skewchar\tenmi ='177  %  for placement of accents
%\font\tenmib = cmmib10   % bold math italic
 \font\tensy  = cmsy10   % math symbols
    \skewchar\tensy ='60 %   for placement of math accents
%\font\tensyb = cmbsy10  % bold symbols
 \font\tenit  = ptmri   % text italic
 \font\tensl  = ptmro   % slanted
 \font\tenbf  = ptmb   % extended bold
%\font\tenbfs = ptmbo % extended bold slanted
 \font\tentt  = pcrb   % typewriter
    \hyphenchar\tentt = -1         %  suppress hyphenation in \tt font
%\font\tentti = cmitt10  % italic typewriter
%\font\tentts = pcrbo % slanted typewriter
 \font\tensf  = phvr at 10pt   % sans serif
%\font\tensfi = cmssi10  % italic sans serif
%\font\tensfb = phvb at 10pt % bold sans serif
%\font\tensc  = ptmrc  % small caps
 \font\tenly  = lasy10  % LaTeX symbols
%\font\tenlyb = lasyb10 % bold LaTeX symbols
%\font\tenuit = ptmru    % unslanted italic

% eleven point
 \font\elvrm  = ptmr at 11pt      % roman
 \font\elvmi  = cmmi10   \@halfmag % math italic
    \skewchar\elvmi ='177          %  for placement of accents
%\font\elvmib = cmmib10   \@halfmag % bold math italic
 \font\elvsy  = cmsy10   \@halfmag % math symbols
    \skewchar\elvsy ='60           %   for placement of math accents
%\font\elvsyb = cmbsy10  \@halfmag % bold symbols
 \font\elvit  = ptmri at 11pt      % text italic
 \font\elvsl  = ptmro at 11pt      % slanted
 \font\elvbf  = ptmb at 11pt      % exelvded bold
%\font\elvbfs = ptmbo at 11pt   % exelvded bold slanted
 \font\elvtt  = pcrb at 11pt      % typewriter
    \hyphenchar\elvtt = -1         %  suppress hyphenation in \tt font
%\font\elvtti = cmitt10  \@halfmag % italic typewriter
%\font\elvtts = pcrbo at 11pt      % slanted typewriter
 \font\elvsf  = phvr at 11pt      % sans serif
%\font\elvsfi = phvro at 11pt      % italic sans serif
%\font\elvsfb = phvb at 11pt      % bold sans serif
%\font\elvsc  = ptmrc at 11pt    % small caps
 \font\elvly  = lasy10  \@halfmag % LaTeX symbols
%\font\elvlyb = lasyb10 \@halfmag % bold LaTeX symbols
%\font\elvuit = ptmru at 11pt    % unslanted italic

% twelve point
 \font\twlrm  = ptmr at 12pt         % roman
 \font\twlmi  = cmmi12               % math italic
    \skewchar\twlmi ='177          %  for placement of accents
%\font\twlmib = cmmib10   \@magscale1 % bold math italic
 \font\twlsy  = cmsy10   \@magscale1 % math symbols
    \skewchar\twlsy ='60           %   for placement of math accents
%\font\twlsyb = cmbsy10  \@magscale1 % bold symbols
 \font\twlit  = ptmri at 12pt        % text italic
 \font\twlsl  = ptmro at 12pt        % slanted
 \font\twlbf  = ptmb at 12pt        % extended bold
%\font\twlbfs = ptmbo at 12pt     % extended bold slanted
 \font\twltt  = pcrb at 12pt        % typewriter
    \hyphenchar\twltt = -1         %  suppress hyphenation in \tt font
%\font\twltti = cmitt10  \@magscale1 % italic typewriter
%\font\twltts = pcrbo at 12pt        % slanted typewriter
 \font\twlsf  = phvr at 12pt        % sans serif
%\font\twlsfi = cmssi10  \@magscale1 % italic sans serif
%\font\twlsfb = phvb at 12pt        % bold sans serif
%\font\twlsc  = ptmrc at 12pt      % small caps
 \font\twlly  = lasy10  \@magscale1 % LaTeX symbols
%\font\twllyb = lasyb10 \@magscale1 % bold LaTeX symbols
%\font\twluit = ptmru at 12pt      % unslanted italic

% fourteen point
 \font\frtnrm  = ptmr at 14pt        % roman
 \font\frtnmi  = cmmi10   \@magscale2 % math italic
    \skewchar\frtnmi ='177          %  for placement of accents
%\font\frtnmib = cmmib10   \@magscale2 % bold math italic
 \font\frtnsy  = cmsy10   \@magscale2 % math symbols
    \skewchar\frtnsy ='60           %   for placement of math accents
%\font\frtnsyb = cmbsy10  \@magscale2 % bold symbols
%\font\frtnit  = ptmri at 14pt        % text italic
%\font\frtnsl  = ptmro at 14pt        % slanted
 \font\frtnbf  = ptmb at 14pt        % extended bold
%\font\frtnbfs = ptmbo at 14pt     % extended bold slanted
%\font\frtntt  = pcrb at 14pt        % typewriter
%\font\frtntti = cmitt10  \@magscale2 % italic typewriter
%\font\frtntts = pcrbo at 14pt        % slanted typewriter
%\font\frtnsf  = phvr at 14pt        % sans serif
%\font\frtnsfi = cmssi10  \@magscale2 % italic sans serif
%\font\frtnsfb = phvb at 14pt        % bold sans serif
%\font\frtnsc  = ptmrc at 14pt      % small caps
 \font\frtnly  = lasy10  \@magscale2 % LaTeX symbols
%\font\frtnlyb = lasyb10 \@magscale2 % bold LaTeX symbols
%\font\frtnuit = ptmru at 14pt      % unslanted italic

% seventeen point
 \font\svtnrm  = ptmr at 17pt        % roman
 \font\svtnmi  = cmmi10   \@magscale3 % math italic
    \skewchar\svtnmi ='177          %  for placement of accents
%\font\svtnmib = cmmib10   \@magscale3 % bold math italic
 \font\svtnsy  = cmsy10   \@magscale3 % math symbols
    \skewchar\svtnsy ='60           %   for placement of math accents
%\font\svtnsyb = cmbsy10  \@magscale3 % bold symbols
%\font\svtnit  = ptmri at 17pt        % text italic
%\font\svtnsl  = ptmro at 17pt        % slanted
 \font\svtnbf  = ptmb at 17pt        % extended bold
%\font\svtnbfs = ptmbo at 17pt     % extended bold slanted
%\font\svtntt  = pcrb at 17pt        % typewriter
%\font\svtntti = cmitt10  \@magscale3 % italic typewriter
%\font\svtntts = pcrbo at 17pt        % slanted typewriter
%\font\svtnsf  = phvr at 17pt        % sans serif
%\font\svtnsfi = cmssi10  \@magscale3 % italic sans serif
%\font\svtnsfb = phvb at 17pt        % bold sans serif
%\font\svtnsc  = ptmrc at 17pt      % small caps
 \font\svtnly  = lasy10  \@magscale3 % LaTeX symbols
%\font\svtnlyb = lasyb10 \@magscale3 % bold LaTeX symbols
%\font\svtnuit = ptmru at 17pt      % unslanted italic

% twenty point
 \font\twtyrm  = ptmr at 20pt        % roman
 \font\twtymi  = cmmi10   \@magscale4 % math italic
    \skewchar\twtymi ='177          %  for placement of accents
%\font\twtymib = cmmib10   \@magscale4 % bold math italic
 \font\twtysy  = cmsy10   \@magscale4 % math symbols
    \skewchar\twtysy ='60           %   for placement of math accents
%\font\twtysyb = cmbsy10  \@magscale4 % bold symbols
%\font\twtyit  = ptmri at 20pt        % text italic
%\font\twtysl  = ptmro at 20pt        % slanted
%\font\twtybf  = ptmb at 20pt        % extended bold
%\font\twtybfs = ptmbo at 20pt     % extended bold slanted
%\font\twtytt  = pcrb at 20pt        % typewriter
%\font\twtytti = cmitt10  \@magscale4 % italic typewriter
%\font\twtytts = pcrbo at 20pt        % slanted typewriter
%\font\twtysf  = phvr at 20pt        % sans serif
%\font\twtysfi = cmssi10  \@magscale4 % italic sans serif
%\font\twtysfb = phvb at 20pt        % bold sans serif
%\font\twtysc  = ptmrc at 20pt      % small caps
 \font\twtyly  = lasy10  \@magscale4 % LaTeX symbols
%\font\twtylyb = lasyb10 \@magscale4 % bold LaTeX symbols
%\font\twtyuit = ptmru at 20pt      % unslanted italic

% twenty-five point
 \font\twfvrm  = ptmr at 25pt        % roman
%\font\twfvmi  = cmmi10   \@magscale5 % math italic
%\font\twfvmib = cmmib10   \@magscale5 % bold math italic
%\font\twfvsy  = cmsy10   \@magscale5 % math symbols
%\font\twfvsyb = cmbsy10  \@magscale5 % bold symbols
%\font\twfvit  = ptmri at 25pt        % text italic
%\font\twfvsl  = ptmro at 25pt        % slanted
%\font\twfvbf  = ptmb at 25pt        % extended bold
%\font\twfvbfs = ptmbo at 25pt     % extended bold slanted
%\font\twfvtt  = pcrb at 25pt        % typewriter
%\font\twfvtti = cmitt10  \@magscale5 % italic typewriter
%\font\twfvtts = pcrbo at 25pt        % slanted typewriter
%\font\twfvsf  = phvr at 25pt        % sans serif
%\font\twfvsfi = cmssi10  \@magscale5 % italic sans serif
%\font\twfvsfb = phvb at 25pt        % bold sans serif
%\font\twfvsc  = ptmrc at 25pt      % small caps
%\font\twfvly  = lasy10   \@magscale5 % LaTeX symbols
%\font\twfvlyb = lasyb10  \@magscale5 % bold LaTeX symbols
%\font\twfvuit = ptmru at 25pt      % unslanted italic

% Math extension
 \font\tenex   = cmex10

% line & circle fonts
\font\tenln    = line10
\font\tenlnw   = linew10
\font\tencirc  = lcircle10    % 21 Nov 89 : circle10 -> lcircle10
\font\tencircw = lcirclew10   % 21 Nov 89 : circlew10 -> lcirclew10

% Change made 6 May 86: `\@warning' replaced by `\immediate\write 15'
% since it's not defined when lfonts.tex is read by lplain.tex.
%
\ifnum\fontdimen8\tenln=\fontdimen8\tencirc \else
  \immediate\write 15{Incompatible thin line and circle fonts}\fi
\ifnum\fontdimen8\tenlnw=\fontdimen8\tencircw \else
  \immediate\write 15{Incompatible thick line and circle fonts}\fi


% protected font names
\def\rm{\protect\prm}
\def\it{\protect\pit}
\def\bf{\protect\pbf}
\def\sl{\protect\psl}
\def\sf{\protect\psf}
\def\sc{\protect\psc}
\def\tt{\protect\ptt}

%% FONT-CUSTOMIZING:  The following definitions define certain commands
%% to be abbreviations for certain font names.  These commands are used
%% below in \@getfont commands, which load the loaded-on-demand fonts.
%% This is done only to save space.  To change the fonts that are loaded
%% on demand, one can either change these definitions or else change
%% the arguments of the \@getfont commands.
%%
\def\@mbi{cmmib10}
\def\@mbsy{cmbsy10}
\def\@mcsc{ptmrc}
\def\@mss{phvr}
\def\@lasyb{lasyb10}

% families

\newfam\itfam      % \it is family 4
\newfam\slfam      % \sl is family 5
\newfam\bffam      % \bf is family 6
\newfam\ttfam      % \tt is family 7
\newfam\sffam      % \sf is family 8
\newfam\scfam      % \sf is family 9
\newfam\lyfam      % \ly is family 10

\def\cal{\fam\tw@}
\def\mit{\fam\@ne}

\def\@setstrut{\setbox\strutbox=\hbox{\vrule \@height .7\baselineskip
    \@depth .3\baselineskip \@width\z@}}


%% FONT-CUSTOMIZING: The commands \vpt, \vipt, ... , \xxvpt perform all
%% the declarations needed to change the type size to 5pt, 6pt, ... ,
%% 25pt.  To see how this works, consider the definition of \viipt,
%% which determines the fonts used in a 7pt type size.  The command
%%    \def\pit{\fam\itfam\sevit}
%% means that the \it command causes the preloaded \sevit font to
%% be used--this font was defined earlier with a \font\sevit...
%% command.  The commands
%%     \textfont\itfam\sevit
%%     \scriptfont\itfam\sevit
%%     \scriptscriptfont\itfam\sevit
%% tell TeX to use the \sevit font for all three math-mode sizes
%% (text, script, and scriptscript) for the 7pt size.
%% The fonts appearing in these commands must be preloaded.
%%
%% The command
%%     \def\pbf{\@getfont\pbf\bffam\@viipt{ambx7}}
%% declares \bf to use a loaded-on-demand font--namely, the font
%% ambx7.
%%
%% The command
%%     \def\ptt{\@subfont\tt\rm}
%% declares that the \tt font is unavailable in the 7pt size, so
%% the \rm font is used instead.  (The substituted type style should
%% correspond to a preloaded size.)

\def\vpt{\textfont\z@\fivrm
  \scriptfont\z@\fivrm \scriptscriptfont\z@\fivrm
\textfont\@ne\fivmi \scriptfont\@ne\fivmi \scriptscriptfont\@ne\fivmi
\textfont\tw@\fivsy \scriptfont\tw@\fivsy \scriptscriptfont\tw@\fivsy
\textfont\thr@@\tenex \scriptfont\thr@@\tenex \scriptscriptfont\thr@@\tenex
\def\prm{\fam\z@\fivrm}%
\def\unboldmath{\everymath{}\everydisplay{}\@nomath
  \unboldmath\fam\@ne\@boldfalse}\@boldfalse
\def\boldmath{\@subfont\boldmath\unboldmath}%
\def\pit{\@subfont\it\rm}%
\def\psl{\@subfont\sl\rm}%
\def\pbf{\@getfont\pbf\bffam\@vpt{ptmb at 5pt}}%
\def\ptt{\@subfont\tt\rm}%
\def\psf{\@subfont\sf\rm}%
\def\psc{\@subfont\sc\rm}%
\def\ly{\fam\lyfam\fivly}\textfont\lyfam\fivly
    \scriptfont\lyfam\fivly \scriptscriptfont\lyfam\fivly
\@setstrut\rm}

\def\@vpt{}

\def\vipt{\textfont\z@\sixrm
  \scriptfont\z@\sixrm \scriptscriptfont\z@\sixrm
\textfont\@ne\sixmi \scriptfont\@ne\sixmi \scriptscriptfont\@ne\sixmi
\textfont\tw@\sixsy \scriptfont\tw@\sixsy \scriptscriptfont\tw@\sixsy
\textfont\thr@@\tenex \scriptfont\thr@@\tenex \scriptscriptfont\thr@@\tenex
\def\prm{\fam\z@\sixrm}%
\def\unboldmath{\everymath{}\everydisplay{}\@nomath
  \unboldmath\@boldfalse}\@boldfalse
\def\boldmath{\@subfont\boldmath\unboldmath}%
\def\pit{\@subfont\it\rm}%
\def\psl{\@subfont\sl\rm}%
\def\pbf{\@getfont\pbf\bffam\@vipt{ptmb at 6pt}}%
\def\ptt{\@subfont\tt\rm}%
\def\psf{\@subfont\sf\rm}%
\def\psc{\@subfont\sc\rm}%
\def\ly{\fam\lyfam\sixly}\textfont\lyfam\sixly
    \scriptfont\lyfam\sixly \scriptscriptfont\lyfam\sixly
\@setstrut\rm}

\def\@vipt{}

\def\viipt{\textfont\z@\sevrm
  \scriptfont\z@\sixrm \scriptscriptfont\z@\fivrm
\textfont\@ne\sevmi \scriptfont\@ne\fivmi \scriptscriptfont\@ne\fivmi
\textfont\tw@\sevsy \scriptfont\tw@\fivsy \scriptscriptfont\tw@\fivsy
\textfont\thr@@\tenex \scriptfont\thr@@\tenex \scriptscriptfont\thr@@\tenex
\def\prm{\fam\z@\sevrm}%
\def\unboldmath{\everymath{}\everydisplay{}\@nomath
\unboldmath\@boldfalse}\@boldfalse
\def\boldmath{\@subfont\boldmath\unboldmath}%
\def\pit{\fam\itfam\sevit}\textfont\itfam\sevit
   \scriptfont\itfam\sevit \scriptscriptfont\itfam\sevit
\def\psl{\@subfont\sl\it}%
\def\pbf{\@getfont\pbf\bffam\@viipt{ptmb at 7pt}}%
\def\ptt{\@subfont\tt\rm}%
\def\psf{\@subfont\sf\rm}%
\def\psc{\@subfont\sc\rm}%
\def\ly{\fam\lyfam\sevly}\textfont\lyfam\sevly
    \scriptfont\lyfam\fivly \scriptscriptfont\lyfam\fivly
\@setstrut \rm}

\def\@viipt{}

\def\viiipt{\textfont\z@\egtrm
  \scriptfont\z@\sixrm \scriptscriptfont\z@\fivrm
\textfont\@ne\egtmi \scriptfont\@ne\sixmi \scriptscriptfont\@ne\fivmi
\textfont\tw@\egtsy \scriptfont\tw@\sixsy \scriptscriptfont\tw@\fivsy
\textfont\thr@@\tenex \scriptfont\thr@@\tenex \scriptscriptfont\thr@@\tenex
\def\prm{\fam\z@\egtrm}%
\def\unboldmath{\everymath{}\everydisplay{}\@nomath
\unboldmath\@boldfalse}\@boldfalse
\def\boldmath{\@subfont\boldmath\unboldmath}%
\def\pit{\fam\itfam\egtit}\textfont\itfam\egtit
   \scriptfont\itfam\sevit \scriptscriptfont\itfam\sevit
\def\psl{\@getfont\psl\slfam\@viiipt{ptmro at 8pt}}%
\def\pbf{\@getfont\pbf\bffam\@viiipt{ptmbo at 8pt}}%
\def\ptt{\@getfont\ptt\ttfam\@viiipt{pcrb at 8pt}\@nohyphens\ptt\@viiipt}%
\def\psf{\@getfont\psf\sffam\@viiipt{phvr at 8pt}}%
\def\psc{\@getfont\psc\scfam\@viiipt{\@mcsc \@ptscale8}}%
\def\ly{\fam\lyfam\egtly}\textfont\lyfam\egtly
    \scriptfont\lyfam\sixly \scriptscriptfont\lyfam\fivly
\@setstrut \rm}

\def\@viiipt{}

\def\ixpt{\textfont\z@\ninrm
  \scriptfont\z@\sixrm \scriptscriptfont\z@\fivrm
\textfont\@ne\ninmi \scriptfont\@ne\sixmi \scriptscriptfont\@ne\fivmi
\textfont\tw@\ninsy \scriptfont\tw@\sixsy \scriptscriptfont\tw@\fivsy
\textfont\thr@@\tenex \scriptfont\thr@@\tenex \scriptscriptfont\thr@@\tenex
\def\prm{\fam\z@\ninrm}%
\def\unboldmath{\everymath{}\everydisplay{}\@nomath\unboldmath
    \@boldfalse}\@boldfalse
\def\boldmath{\@subfont\boldmath\unboldmath}%
\def\pit{\fam\itfam\ninit}\textfont\itfam\ninit
   \scriptfont\itfam\sevit \scriptscriptfont\itfam\sevit
\def\psl{\@getfont\psl\slfam\@ixpt{ptmro at 9pt}}%
\def\pbf{\fam\bffam\ninbf}\textfont\bffam\ninbf
   \scriptfont\bffam\ninbf \scriptscriptfont\bffam\ninbf
\def\ptt{\fam\ttfam\nintt}\textfont\ttfam\nintt
   \scriptfont\ttfam\nintt \scriptscriptfont\ttfam\nintt
\def\psf{\@getfont\psf\sffam\@ixpt{phvr at 9pt}}%
\def\psc{\@getfont\psc\scfam\@ixpt{\@mcsc \@ptscale9}}%
\def\ly{\fam\lyfam\ninly}\textfont\lyfam\ninly
   \scriptfont\lyfam\sixly \scriptscriptfont\lyfam\fivly
\@setstrut \rm}

\def\@ixpt{}

\def\xpt{\textfont\z@\tenrm
  \scriptfont\z@\sevrm \scriptscriptfont\z@\fivrm
\textfont\@ne\tenmi \scriptfont\@ne\sevmi \scriptscriptfont\@ne\fivmi
\textfont\tw@\tensy \scriptfont\tw@\sevsy \scriptscriptfont\tw@\fivsy
\textfont\thr@@\tenex \scriptfont\thr@@\tenex \scriptscriptfont\thr@@\tenex
\def\unboldmath{\everymath{}\everydisplay{}\@nomath\unboldmath
          \textfont\@ne\tenmi
          \textfont\tw@\tensy \textfont\lyfam\tenly
          \@boldfalse}\@boldfalse
\def\boldmath{\@ifundefined{tenmib}{\global\font\tenmib\@mbi
   \global\font\tensyb\@mbsy
   \global\font\tenlyb\@lasyb\relax\@addfontinfo\@xpt
   {\def\boldmath{\everymath{\mit}\everydisplay{\mit}\@prtct\@nomathbold
        \textfont\@ne\tenmib \textfont\tw@\tensyb
        \textfont\lyfam\tenlyb \@prtct\@boldtrue}}}{}\@xpt\boldmath}%
\def\prm{\fam\z@\tenrm}%
\def\pit{\fam\itfam\tenit}\textfont\itfam\tenit \scriptfont\itfam\sevit
    \scriptscriptfont\itfam\sevit
\def\psl{\fam\slfam\tensl}\textfont\slfam\tensl
     \scriptfont\slfam\tensl \scriptscriptfont\slfam\tensl
\def\pbf{\fam\bffam\tenbf}\textfont\bffam\tenbf
    \scriptfont\bffam\tenbf \scriptscriptfont\bffam\tenbf
\def\ptt{\fam\ttfam\tentt}\textfont\ttfam\tentt
    \scriptfont\ttfam\tentt \scriptscriptfont\ttfam\tentt
\def\psf{\fam\sffam\tensf}\textfont\sffam\tensf
    \scriptfont\sffam\tensf \scriptscriptfont\sffam\tensf
\def\psc{\@getfont\psc\scfam\@xpt{\@mcsc}}%
\def\ly{\fam\lyfam\tenly}\textfont\lyfam\tenly
   \scriptfont\lyfam\sevly \scriptscriptfont\lyfam\fivly
\@setstrut \rm}

\def\@xpt{}

\def\xipt{\textfont\z@\elvrm
  \scriptfont\z@\egtrm \scriptscriptfont\z@\sixrm
\textfont\@ne\elvmi \scriptfont\@ne\egtmi \scriptscriptfont\@ne\sixmi
\textfont\tw@\elvsy \scriptfont\tw@\egtsy \scriptscriptfont\tw@\sixsy
\textfont\thr@@\tenex \scriptfont\thr@@\tenex \scriptscriptfont\thr@@\tenex
\def\unboldmath{\everymath{}\everydisplay{}\@nomath\unboldmath
          \textfont\@ne\elvmi \textfont\tw@\elvsy
          \textfont\lyfam\elvly \@boldfalse}\@boldfalse
\def\boldmath{\@ifundefined{elvmib}{\global\font\elvmib\@mbi\@halfmag
         \global\font\elvsyb\@mbsy\@halfmag
         \global\font\elvlyb\@lasyb\@halfmag\relax\@addfontinfo\@xipt
         {\def\boldmath{\everymath{\mit}\everydisplay{\mit}\@prtct\@nomathbold
                \textfont\@ne\elvmib \textfont\tw@\elvsyb
                \textfont\lyfam\elvlyb\@prtct\@boldtrue}}}{}\@xipt\boldmath}%
\def\prm{\fam\z@\elvrm}%
\def\pit{\fam\itfam\elvit}\textfont\itfam\elvit
   \scriptfont\itfam\egtit \scriptscriptfont\itfam\sevit
\def\psl{\fam\slfam\elvsl}\textfont\slfam\elvsl
    \scriptfont\slfam\tensl \scriptscriptfont\slfam\tensl
\def\pbf{\fam\bffam\elvbf}\textfont\bffam\elvbf
   \scriptfont\bffam\ninbf \scriptscriptfont\bffam\ninbf
\def\ptt{\fam\ttfam\elvtt}\textfont\ttfam\elvtt
   \scriptfont\ttfam\nintt \scriptscriptfont\ttfam\nintt
\def\psf{\fam\sffam\elvsf}\textfont\sffam\elvsf
    \scriptfont\sffam\tensf \scriptscriptfont\sffam\tensf
\def\psc{\@getfont\psc\scfam\@xipt{\@mcsc\@halfmag}}%
\def\ly{\fam\lyfam\elvly}\textfont\lyfam\elvly
   \scriptfont\lyfam\egtly \scriptscriptfont\lyfam\sixly
\@setstrut \rm}

\def\@xipt{}

\def\xiipt{\textfont\z@\twlrm
  \scriptfont\z@\egtrm \scriptscriptfont\z@\sixrm
\textfont\@ne\twlmi \scriptfont\@ne\egtmi \scriptscriptfont\@ne\sixmi
\textfont\tw@\twlsy \scriptfont\tw@\egtsy \scriptscriptfont\tw@\sixsy
\textfont\thr@@\tenex \scriptfont\thr@@\tenex \scriptscriptfont\thr@@\tenex
\def\unboldmath{\everymath{}\everydisplay{}\@nomath\unboldmath
          \textfont\@ne\twlmi
          \textfont\tw@\twlsy \textfont\lyfam\twlly
          \@boldfalse}\@boldfalse
\def\boldmath{\@ifundefined{twlmib}{\global\font\twlmib\@mbi\@magscale1\global
        \font\twlsyb\@mbsy \@magscale1\global\font
         \twllyb\@lasyb\@magscale1\relax\@addfontinfo\@xiipt
              {\def\boldmath{\everymath
                {\mit}\everydisplay{\mit}\@prtct\@nomathbold
                \textfont\@ne\twlmib \textfont\tw@\twlsyb
                \textfont\lyfam\twllyb\@prtct\@boldtrue}}}{}\@xiipt\boldmath}%
\def\prm{\fam\z@\twlrm}%
\def\pit{\fam\itfam\twlit}\textfont\itfam\twlit \scriptfont\itfam\egtit
   \scriptscriptfont\itfam\sevit
\def\psl{\fam\slfam\twlsl}\textfont\slfam\twlsl
     \scriptfont\slfam\tensl \scriptscriptfont\slfam\tensl
\def\pbf{\fam\bffam\twlbf}\textfont\bffam\twlbf
   \scriptfont\bffam\ninbf \scriptscriptfont\bffam\ninbf
\def\ptt{\fam\ttfam\twltt}\textfont\ttfam\twltt
   \scriptfont\ttfam\nintt \scriptscriptfont\ttfam\nintt
\def\psf{\fam\sffam\twlsf}\textfont\sffam\twlsf
    \scriptfont\sffam\tensf \scriptscriptfont\sffam\tensf
\def\psc{\@getfont\psc\scfam\@xiipt{\@mcsc\@magscale1}}%
\def\ly{\fam\lyfam\twlly}\textfont\lyfam\twlly
   \scriptfont\lyfam\egtly \scriptscriptfont\lyfam\sixly
 \@setstrut \rm}

\def\@xiipt{}

\def\xivpt{\textfont\z@\frtnrm
  \scriptfont\z@\tenrm \scriptscriptfont\z@\sevrm
\textfont\@ne\frtnmi \scriptfont\@ne\tenmi \scriptscriptfont\@ne\sevmi
\textfont\tw@\frtnsy \scriptfont\tw@\tensy \scriptscriptfont\tw@\sevsy
\textfont\thr@@\tenex \scriptfont\thr@@\tenex \scriptscriptfont\thr@@\tenex
\def\unboldmath{\everymath{}\everydisplay{}\@nomath\unboldmath
          \textfont\@ne\frtnmi \textfont\tw@\frtnsy
          \textfont\lyfam\frtnly \@boldfalse}\@boldfalse
\def\boldmath{\@ifundefined{frtnmib}{\global\font
        \frtnmib\@mbi\@magscale2\global\font\frtnsyb\@mbsy\@magscale2
         \global\font\frtnlyb\@lasyb\@magscale2\relax\@addfontinfo\@xivpt
               {\def\boldmath{\everymath
                {\mit}\everydisplay{\mit}\@prtct\@nomathbold
              \textfont\@ne\frtnmib \textfont\tw@\frtnsyb
              \textfont\lyfam\frtnlyb\@prtct\@boldtrue}}}{}\@xivpt\boldmath}%
\def\prm{\fam\z@\frtnrm}%
\def\pit{\@getfont\pit\itfam\@xivpt{ptmri at 14pt}}%
\def\psl{\@getfont\psl\slfam\@xivpt{ptmro at 14pt}}%
\def\pbf{\fam\bffam\frtnbf}\textfont\bffam\frtnbf
   \scriptfont\bffam\tenbf \scriptscriptfont\bffam\ninbf
\def\ptt{\@getfont\ptt\ttfam\@xivpt{pcrb at 14pt}\@nohyphens\ptt\@xivpt}%
\def\psf{\@getfont\psf\sffam\@xivpt{\@mss\@magscale2}}%
\def\psc{\@getfont\psc\scfam\@xivpt{\@mcsc\@magscale2}}%
\def\ly{\fam\lyfam\frtnly}\textfont\lyfam\frtnly
   \scriptfont\lyfam\tenly \scriptscriptfont\lyfam\sevly
\@setstrut \rm}

\def\@xivpt{}

\def\xviipt{\textfont\z@\svtnrm
  \scriptfont\z@\twlrm \scriptscriptfont\z@\tenrm
\textfont\@ne\svtnmi \scriptfont\@ne\twlmi \scriptscriptfont\@ne\tenmi
\textfont\tw@\svtnsy \scriptfont\tw@\twlsy \scriptscriptfont\tw@\tensy
\textfont\thr@@\tenex \scriptfont\thr@@\tenex \scriptscriptfont\thr@@\tenex
\def\unboldmath{\everymath{}\everydisplay{}\@nomath\unboldmath
          \textfont\@ne\svtnmi \textfont\tw@\svtnsy \textfont\lyfam\svtnly
          \@boldfalse}\@boldfalse
\def\boldmath{\@subfont\boldmath\unboldmath}%
\def\prm{\fam\z@\svtnrm}%
\def\pit{\@getfont\pit\itfam\@xviipt{ptmri at 17pt}}%
\def\psl{\@getfont\psl\slfam\@xviipt{ptmro at 17pt}}%
\def\pbf{\fam\bffam\svtnbf}\textfont\bffam\svtnbf
    \scriptfont\bffam\twlbf \scriptscriptfont\bffam\tenbf
\def\ptt{\@getfont\ptt\ttfam\@xviipt{pcrb at 17pt}\@nohyphens
   \ptt\@xviipt}%
\def\psf{\@getfont\psf\sffam\@xviipt{\@mss\@magscale3}}%
\def\psc{\@getfont\psc\scfam\@xviipt{\@mcsc\@magscale3}}%
\def\ly{\fam\lyfam\svtnly}\textfont\lyfam\svtnly
   \scriptfont\lyfam\twlly   \scriptscriptfont\lyfam\tenly
\@setstrut \rm}

\def\@xviipt{}

\def\xxpt{\textfont\z@\twtyrm
  \scriptfont\z@\frtnrm \scriptscriptfont\z@\twlrm
\textfont\@ne\twtymi \scriptfont\@ne\frtnmi \scriptscriptfont\@ne\twlmi
\textfont\tw@\twtysy \scriptfont\tw@\frtnsy \scriptscriptfont\tw@\twlsy
\textfont\thr@@\tenex \scriptfont\thr@@\tenex \scriptscriptfont\thr@@\tenex
\def\unboldmath{\everymath{}\everydisplay{}\@nomath\unboldmath
        \textfont\@ne\twtymi \textfont\tw@\twtysy \textfont\lyfam\twtyly
        \@boldfalse}\@boldfalse
\def\boldmath{\@subfont\boldmath\unboldmath}%
\def\prm{\fam\z@\twtyrm}%
\def\pit{\@getfont\pit\itfam\@xxpt{ptmri at 20pt}}%
\def\psl{\@getfont\psl\slfam\@xxpt{ptmro at 20pt}}%
\def\pbf{\@getfont\pbf\bffam\@xxpt{ptmb at 20pt}}%
\def\ptt{\@getfont\ptt\ttfam\@xxpt{pcrb at 20pt}\@nohyphens\ptt\@xxpt}%
\def\psf{\@getfont\psf\sffam\@xxpt{\@mss\@magscale4}}%
\def\psc{\@getfont\psc\scfam\@xxpt{\@mcsc\@magscale4}}%
\def\ly{\fam\lyfam\twtyly}\textfont\lyfam\twtyly
   \scriptfont\lyfam\frtnly \scriptscriptfont\lyfam\twlly
\@setstrut \rm}

\def\@xxpt{}

\def\xxvpt{\textfont\z@\twfvrm
  \scriptfont\z@\twtyrm \scriptscriptfont\z@\svtnrm
\textfont\@ne\twtymi \scriptfont\@ne\twtymi \scriptscriptfont\@ne\svtnmi
\textfont\tw@\twtysy \scriptfont\tw@\twtysy \scriptscriptfont\tw@\svtnsy
\textfont\thr@@\tenex \scriptfont\thr@@\tenex \scriptscriptfont\thr@@\tenex
\def\unboldmath{\everymath{}\everydisplay{}\@nomath\unboldmath
        \textfont\@ne\twtymi \textfont\tw@\twtysy \textfont\lyfam\twtyly
        \@boldfalse}\@boldfalse
\def\boldmath{\@subfont\boldmath\unboldmath}%
\def\prm{\fam\z@\twfvrm}%
\def\pit{\@subfont\it\rm}%
\def\psl{\@subfont\sl\rm}%
\def\pbf{\@getfont\pbf\bffam\@xxvpt{ptmb at 25pt}}%
\def\ptt{\@subfont\tt\rm}%
\def\psf{\@subfont\sf\rm}%
\def\psc{\@subfont\sc\rm}%
\def\ly{\fam\lyfam\twtyly}\textfont\lyfam\twtyly
   \scriptfont\lyfam\twtyly \scriptscriptfont\lyfam\svtnly
\@setstrut \rm}

\def\@xxvpt{}

% SPECIAL LaTeX character definitions

% Definitions of math operators added by LaTeX
\mathchardef\mho"0A30
\mathchardef\Join"3A31
\mathchardef\Box"0A32
\mathchardef\Diamond"0A33
\mathchardef\leadsto"3A3B
\mathchardef\sqsubset"3A3C
\mathchardef\sqsupset"3A3D
\def\lhd{\mathbin{< \hbox to -.43em{}\hbox{\vrule
      \@width .065em \@height .55em \@depth .05em}\hbox to .2em{}}}
\def\rhd{\mathbin{\hbox to .3em{}\hbox{\vrule \@width .065em \@height
       .55em \@depth .05em}\hbox to -.43em{}>}}
\def\unlhd{\mathbin{\leq \hbox to -.43em{}\hbox
        {\vrule \@width .065em \@height .63em \@depth -.08em}\hbox to .2em{}}}
\def\unrhd{\mathbin{ \hbox to .3em{}\hbox
 {\vrule \@width .065em \@height .63em \@depth -.08em}\hbox to -.43em{}\geq}}

% Definition of \$ to work in italic font (since it produces a pound sterling
% sign in the cmit font.

\def\${\protect\pdollar}
\def\pdollar{{\ifdim \fontdimen\@ne\font >\z@ \sl \fi\char`\$}}

% Definition of pound sterling sign.
% Modified 10 Apr 89 to work in math mode.

\def\pounds{\protect\ppounds}
\def\ppounds{\relax\ifmmode\mathchar"424\else{\it \char'44}\fi}


% Definition of \copyright changed so it works in other type styles,
% and so it is robust
\def\copyright{\protect\pcopyright}
\def\pcopyright{{\rm\ooalign{\hfil
     \raise.07ex\hbox{c}\hfil\crcr\mathhexbox20D}}}
