% macros for index produced by twinx

\let\:=\. % preserve a way to get the dot accent
 % (all other accents will still work as usual)
\def\TeX{{\ifmmode\it\fi
   \hbox{T\kern-.1667em\lower.424ex\hbox{E}\hskip-.125em X}}}
\def\,{\relax\ifmmode\mskip\thinmuskip\else\thinspace\fi}
\let\mc=\ninerm % medium caps
\def\CEE/{{\mc C\spacefactor1000}}
\def\UNIX/{{\mc U\kern-.05emNIX\spacefactor1000}}
\def\TEX/{\TeX}
\def\CPLUSPLUS/{{\mc C\PP\spacefactor1000}}
\def\9#1{}

\newdimen\em \em=10pt % this "em" will not change with font size
\parskip 0pt plus .1pt % almost no stretch between paragraphs
\parindent 1\em % for paragraphs and for the first line of C text

\font\ninerm=cmr9
\font\eightrm=cmr8
\font\sixrm=cmr6
\font\ninei=cmmi9
\font\eighti=cmmi8
\font\sixi=cmmi6
\skewchar\ninei='177 \skewchar\eighti='177 \skewchar\sixi='177
\font\ninesy=cmsy9
\font\eightsy=cmsy8
\font\sixsy=cmsy6
\skewchar\ninesy='60 \skewchar\eightsy='60 \skewchar\sixsy='60
\font\ninebf=cmbx9
\font\eightbf=cmbx8
\font\sixbf=cmbx6
\font\ninett=cmtt9
\font\eighttt=cmtt8
\hyphenchar\ninett=-1 \hyphenchar\eighttt=-1
\font\ninesl=cmsl9
\font\eightsl=cmsl8
\font\nineit=cmti9
\font\eightit=cmti8
\font\tentex=cmtex10
\font\ninetex=cmtex9 % TeX extended character set (used in strings)
\font\eighttex=cmtex8
\fontdimen7\tentex=0pt % no double space after sentences
\fontdimen7\ninetex=0pt
\fontdimen7\eighttex=0pt

\newif\iftenpoint
\def\tenpoint{\tenpointtrue
 \def\rm{\fam0\tenrm}%
  \textfont0=\tenrm \scriptfont0=\sevenrm \scriptscriptfont0=\fiverm
  \textfont1=\teni \scriptfont1=\seveni \scriptscriptfont1=\fivei
  \textfont2=\tensy \scriptfont2=\sevensy \scriptscriptfont2=\fivesy
  \textfont3=\tenex \scriptfont3=\tenex \scriptscriptfont3=\tenex
  \def\it{\fam\itfam\tenit}%
  \textfont\itfam=\tenit
  \def\sl{\fam\slfam\tensl}%
  \textfont\slfam=\tensl
  \def\bf{\fam\bffam\tenbf}%
  \textfont\bffam=\tenbf \scriptfont\bffam=\sevenbf
   \scriptscriptfont\bffam=\fivebf
  \def\tt{\fam\ttfam\tentt}%
  \textfont\ttfam=\tentt
  \def\ttx{\tentex}%
  \normalbaselineskip=12pt
  \let\mc=\ninerm
  \let\sc=\eightrm
  \let\big=\tenbig
  \setbox\strutbox=\hbox{\vrule height8pt depth3pt width 0pt}%
  \normalbaselines\rm}

\def\ninepoint{\tenpointfalse
 \def\rm{\fam0\ninerm}%
  \textfont0=\ninerm \scriptfont0=\sixrm \scriptscriptfont0=\fiverm
  \textfont1=\ninei \scriptfont1=\sixi \scriptscriptfont1=\fivei
  \textfont2=\ninesy \scriptfont2=\sixsy \scriptscriptfont2=\fivesy
  \textfont3=\tenex \scriptfont3=\tenex \scriptscriptfont3=\tenex
  \def\it{\fam\itfam\nineit}%
  \textfont\itfam=\nineit
  \def\sl{\fam\slfam\ninesl}%
  \textfont\slfam=\ninesl
  \def\bf{\fam\bffam\ninebf}%
  \textfont\bffam=\ninebf \scriptfont\bffam=\sixbf
   \scriptscriptfont\bffam=\fivebf
  \def\tt{\fam\ttfam\ninett}%
  \textfont\ttfam=\ninett
  \def\ttx{\ninetex}%
  \normalbaselineskip=11pt
  \let\mc=\eightrm
  \let\sc=\sevenrm
  \let\big=\ninebig
  \setbox\strutbox=\hbox{\vrule height8pt depth3pt width 0pt}%
  \normalbaselines\rm}

\def\eightpoint{%
 \def\rm{\fam0\eightrm}%
  \textfont0=\eightrm \scriptfont0=\sixrm \scriptscriptfont0=\fiverm
  \textfont1=\eighti \scriptfont1=\sixi \scriptscriptfont1=\fivei
  \textfont2=\eightsy \scriptfont2=\sixsy \scriptscriptfont2=\fivesy
  \textfont3=\tenex \scriptfont3=\tenex \scriptscriptfont3=\tenex
  \def\it{\fam\itfam\eightit}%
  \textfont\itfam=\eightit
  \def\sl{\fam\slfam\eightsl}%
  \textfont\slfam=\eightsl
  \def\bf{\fam\bffam\eightbf}%
  \textfont\bffam=\eightbf \scriptfont\bffam=\sixbf
   \scriptscriptfont\bffam=\fivebf
  \def\tt{\fam\ttfam\eighttt}%
  \textfont\ttfam=\eighttt
  \def\ttx{\eighttex}%
  \normalbaselineskip=9pt
  \let\mc=\sevenrm
  \let\sc=\sixrm
  \let\big=\eightbig
  \setbox\strutbox=\hbox{\vrule height7pt depth2pt width 0pt}%
  \normalbaselines\rm}

\tenpoint
\def\tenbig#1{{\hbox{$\left#1\vbox to8.5pt{}\right.\nulldelimiterspace=0pt$}}}
\def\ninebig#1{{\hbox{$\textfont0=\tenrm\textfont2=\tensy
  \left#1\vbox to7.25pt{}\right.\nulldelimiterspace=0pt$}}}
\def\eightbig#1{{\hbox{$\textfont0=\ninerm\textfont2=\ninesy
  \left#1\vbox to6.5pt{}\right.\nulldelimiterspace=0pt$}}}

\def\Cee{{\mc C\spacefactor1000}}
\def\UNIX{{\mc UNIX\spacefactor1000}}
\font\titlefont=cmr7 scaled\magstep4 % title on the contents page
\font\ttitlefont=cmtt10 scaled\magstep2 % typewriter type in title

\def\\#1{\leavevmode\hbox{\it#1\/\kern.05em}} % italic type for identifiers
\def\|#1{\leavevmode\hbox{$#1$}} % one-letter identifiers look better this way
\def\&#1{\leavevmode\hbox{\bf
  \def\_{\kern.04em\vbox{\hrule width.3em height .6pt}\kern.08em}%
  #1\/\kern.05em}} % boldface type for reserved words
\def\.#1{\leavevmode\hbox{\ttx % typewriter type for strings
  \let\\=\BS % backslash in a string
  \let\{=\LB % left brace in a string
  \let\}=\RB % right brace in a string
  \let\~=\TL % tilde in a string
  \let\ =\SP % space in a string
  \let\_=\UL % underline in a string
  \let\&=\AM % ampersand in a string
  \let\^=\CF % circumflex in a string
  #1\kern.2em}}
\def\){\discretionary{\hbox{\tentex\BS}}{}{}}
\def\AT{@} % at sign for control text

\chardef\AM=`\& % ampersand character in a string
\chardef\BS=`\\ % backslash in a string
\chardef\LB=`\{ % left brace in a string
\chardef\RB=`\} % right brace in a string
\def\SP{{\tt\char`\ }} % (visible) space in a string
\chardef\TL=`\~ % tilde in a string
\chardef\UL=`\_ % underline character in a string
\chardef\CF=`\^ % circumflex character in a string

\newbox\PPbox % symbol for ++
\setbox\PPbox=\hbox{\kern.5pt\raise1pt\hbox{\sevenrm+\kern-1pt+}\kern.5pt}
\def\PP{\copy\PPbox}
\newbox\MMbox \setbox\MMbox=\hbox{\kern.5pt\raise1pt\hbox{\sevensy\char0
 \kern-1pt\char0}\kern.5pt}
\def\MM{\copy\MMbox}
\newbox\MGbox % symbol for ->
\setbox\MGbox=\hbox{\kern-2pt\lower3pt\hbox{\teni\char'176}\kern1pt}
\def\MG{\copy\MGbox}
\let\GG=\gg
\let\LL=\ll
\let\NULL=\Lambda
\mathchardef\AND="2026 % bitwise and; also \& (unary operator)
\let\OR=\mid % bitwise or
\let\XOR=\oplus % bitwise exclusive or
\def\CM{{\sim}} % bitwise complement
\newbox\MODbox \setbox\MODbox=\hbox{\eightrm\%}
\def\MOD{\mathbin{\copy\MODbox}}

\newbox\bak \setbox\bak=\hbox to -1\em{} % backspace one em
\newbox\bakk\setbox\bakk=\hbox to -2\em{} % backspace two ems

\newcount\ind % current indentation in ems
\def\1{\global\advance\ind by1\hangindent\ind \em} % indent one more notch
\def\2{\global\advance\ind by-1} % indent one less notch
\def\3#1{\hfil\penalty#10\hfilneg} % optional break within a statement
\def\4{\copy\bak} % backspace one notch
\def\5{\hfil\penalty-1\hfilneg\kern2.5\em\copy\bakk\ignorespaces}% optional break
\def\6{\ifmmode\else\par % forced break
  \hangindent\ind \em\noindent\kern\ind \em\copy\bakk\ignorespaces\fi}
\def\7{\Y\6} % forced break and a little extra space
\def\8{\hskip-\ind \em\hskip 2\em} % no indentation

\let\yskip=\smallskip
\def\?{\mathrel?}
\def\note#1#2.{\par\penalty5000
  \Y\noindent{\hangindent2\em\baselineskip10pt\eightrm#1~#2.\par}}
\def\lapstar{\rlap{*}}
\def\stsec{\tenpoint
  \rightskip=0pt % get out of C mode (cf. \B)
  \sfcode`;=1500 \pretolerance 200 \hyphenpenalty 50 \exhyphenpenalty 50
  \noindent\strut{\bf\modno.\quad}}
\let\startsection=\stsec
\def\defin#1{\global\advance\ind by 2 \1\&{#1 } } % begin `define' or `format'
\def\A{\note{See also section}} % xref for doubly defined section name
\def\As{\note{See also sections}} % xref for multiply defined section name
\def\B{\iftenpoint\ninepoint\fi
  \rightskip=0pt plus 100pt minus 10pt % go into C mode
  \sfcode`;=3000
  \pretolerance 10000
  \hyphenpenalty 9999 % so strings can be broken (discretionary \ is inserted)
  \exhyphenpenalty 10000
  \global\ind=2 \1\ \unskip}
\def\C#1{\5\5\quad$/\ast\,$#1$\,\ast/$}
\def\D{\defin{\#define}} % macro definition
\let\E=\equiv % equivalence sign
\def\ET{ and~} % conjunction between two section numbers
\def\ETs{, and~} % conjunction between the last two of several section numbers
\def\F{\defin{format}} % format definition
\let\G=\ge % greater than or equal sign
% \H is long Hungarian umlaut accent
\let\I=\ne % unequal sign
\def\J{\.{@\&}} % TANGLE's join operation
\let\K== % can be changed to left arrow, if desired
% \L is Polish letter suppressed-L; we have no shorthand for \le
% \O is Scandinavian letter O-with-slash
% \P is paragraph sign
% \Q is not used
\let\R=\lnot % logical not
% \S is section sign
\def\T#1{\leavevmode % octal, hex or decimal constant
  \hbox{${\def\?{\kern.2em}%
    \def\${\ell}% long constant
    \def\_{\cdot 10^{\aftergroup}}% power of ten (via dirty trick)
    \let\~=\oct \let\^=\hex #1}$}}
\def\U{\note{This code is used in section}} % xref for use of a section
\def\Us{\note{This code is used in sections}} % xref for uses of a section
\let\V=\lor % logical or
\let\W=\land % logical and
\def\X#1:#2\X{\ifmmode\gdef\XX{\null$\null}\else\gdef\XX{}\fi % section name
  \XX$\langle\,${#2\sevenrm\kern.5em#1}$\,\rangle$\XX}
\def\Y{\par\yskip}
\let\Z=\let % now you can \send the control sequence \Z
\let\*=*

%\def\oct{\hbox{\rm\char'23\kern-.2em\it\aftergroup\?\aftergroup}} % WEB style
%\def\hex{\hbox{\rm\char"7D\tt\aftergroup}} % WEB style
\def\oct{\hbox{$^\circ$\kern-.1em\it\aftergroup\?\aftergroup}}% CWEB style
\def\hex{\hbox{$^{\scriptscriptstyle\#}$\tt\aftergroup}} % CWEB style
\def\vb#1{\leavevmode\hbox{\kern2pt\vrule\vtop{\vbox{\hrule
        \hbox{\strut\kern2pt\.{#1}\kern2pt}}
      \hrule}\vrule\kern2pt}} % verbatim string

\def\normaloutput#1{\shipout\vbox{
      \vbox to 3pc{\ifodd\pageno\rightheadline\else\leftheadline\fi\vfill}
    \nointerlineskip#1}
  \global\advance\pageno 1 }
\def\page{\box255 }

\newif\ifpagesaved \newif\iftitle
\def\leftheadline{\hbox to\pagewd{\vbox to10pt{}%
  \iftitle\global\titlefalse\else\ninesl\rhead\fi
  \hfil\eightrm\folio}}
\def\rightheadline{\hbox to\pagewd{\vbox to10pt{}%
  \eightrm\folio\hfil\ninesl\rhead\/}}
\def\rhead{MASTER INDEX}
\newbox\sbox % saved box preceding the index
\newbox\lbox % lefthand column in the index

\newdimen\pageht \pageht=19cm
\newdimen\pagewd \pagewd=13cm
\newdimen\colwd \colwd=\pagewd
 \advance\colwd by -1pc \divide\colwd by 2 % for two columns

% Usually some text will be inserted at the beginning, preceded by
% \preinx and followed by \inx
\def\preinx{\tenpoint \hsize=\pagewd
  \output{\ifpagesaved\normaloutput{\box\sbox}\fi
    \global\setbox\sbox=\page \global\pagesavedtrue}
  \rightskip0pt \tolerance 200
  \pagesavedfalse \parindent=20pt}
\def\inx{\parindent=0pt
  \vskip 15pt plus 1fil
  \eject\setbox\sbox\vbox{\unvbox\sbox} % take text out of its box
  \vsize=\pageht \advance\vsize by -\ht\sbox % the remaining height
  \rightskip0pt plus 2.5em \tolerance 10000
  \hsize=\colwd
  \parfillskip 0pt plus .6\hsize % try to avoid almost empty lines
  \output{\twocolout} \eightpoint}
\vsize=\pageht
\eightpoint
\def\lr{L} % this tells whether the left or right column is next
\def\twocolout{\if L\lr\global\setbox\lbox=\page \gdef\lr{R}
  \else\normaloutput{\vbox to\pageht{\box\sbox\vss
      \hbox to\pagewd{\box\lbox\hfil\page}}}
  \global\vsize=\pageht\gdef\lr{L}\global\pagesavedfalse\fi}
\output{\twocolout}
\parskip 0pt plus .5pt
\outer\def\I#1\unskip, {\par\hangindent2em\noindent#1:\kern1em} % index entry
\def\[#1]{$\underline{#1}$} % underlined index item
\rm \rightskip0pt plus 2.5em \tolerance 10000
\hyphenpenalty 10000
\parindent0pt
\def\fin{\par\vfill\eject % this is done when we are ending the index
  \ifpagesaved\null\vfill\eject\fi % output a null index column
  \if L\lr\else\null\vfill\eject\fi % finish the current page
  \end}

\input twinx-startup % custom IDs, \preinx copy, etc.

% To produce only a subset of pages, put the page numbers on separate
% lines in a file called pages.tex
\let\Shipout=\shipout
\newread\pages \newcount\nextpage \openin\pages=pages
\def\getnextpage{\ifeof\pages\else
 {\endlinechar=-1\read\pages to\next
  \ifx\next\empty % in this case we should have eof now
  \else\global\nextpage=\next\fi}\fi}
\ifeof\pages\else\message{OK, I'll ship only the requested pages!}
 \getnextpage\fi
\def\shipout{\ifeof\pages\let\next=\Shipout
 \else\ifnum\pageno=\nextpage\getnextpage\let\next=\Shipout
  \else\let\next=\Tosspage\fi\fi \next}
\newbox\garbage \def\Tosspage{\deadcycles=0\setbox\garbage=}
